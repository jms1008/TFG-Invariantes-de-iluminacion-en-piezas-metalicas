\apendice{Plan de Proyecto Software}

\section{Introducción}\label{plan-proyecto-introduccion}

La fase de planificación es esencial en cualquier proyecto. Durante esta etapa, se calculan el trabajo, el tiempo y el dinero necesarios para completar el proyecto. Para ello, se analiza detalladamente cada una de las partes que lo componen, lo que permite conocer los recursos necesarios para su realización. En este anexo se describe todo este proceso.

La fase de planificación se puede dividir en dos partes:

\begin{itemize}
  \item \textbf{La planificación temporal:} Se crea un calendario o cronograma. Aquí se calcula el tiempo necesario para completar cada una de las partes del proyecto. Se debe establecer una fecha de inicio fija y una fecha de finalización estimada, considerando la importancia de cada tarea y los requisitos necesarios para comenzar a trabajar en ellas.
  \item \textbf{El estudio de viabilidad:} Se enfoca en la factibilidad del proyecto y se divide en dos apartados. 
  \begin{itemize}
    \item \textbf{La viabilidad económica:} En esta parte se estiman los costos y los beneficios que podría traer la realización del proyecto. 
    \item \textbf{La viabilidad legal:} Aquí es donde se tiene en cuenta el contexto del proyecto y toca aquella ley que le afecta. Se deben analizar todas aquellas que afecten al proyecto.
  \end{itemize}
\end{itemize}

\section{Planificación temporal}\label{planificacion-temporal}

En líneas generales, se ha seguido el marco de trabajo Scrum, aunque de manera adaptada a la situación de ser el único desarrollador, manteniendo ciertos principios clave.

\begin{itemize}
  \item Se aplicó una estrategia de desarrollo incremental mediante iteraciones (sprints) y revisiones.
  \item La duración media de los sprints ha sido de dos semanas.
  \item Aunque no se han tenido reuniones diarias, se han realizado revisiones periódicas al final de cada sprint para evaluar el progreso y planificar el siguiente.
  \item Esto ha permitido mantener la esencia de la mejora continua y la entrega incremental de valor.
\end{itemize}

Cabe mencionar que no fue hasta la reunión de control del 26 de marzo cuando se empezó a reflejar el progreso en GitHub, añadiendo todas las issues de reuniones anteriores y etiquetándolas con S1, S2 y S3 en el nombre para indicar la semana correspondiente. Por ello, en GitHub todas las issues aparecen como añadidas entre el 26 y el 28 de marzo. Zube se comenzó a utilizar para organizar mejor los sprints a partir del sprint 4, debido a la recomendación de Carlos para tener un mejor control de estos.

\subsection{Sprint 0 (04/03/2024 - 11/03/2024)}\label{sprint-0}

La reunión que dio inicio a este sprint marcó el comienzo del proyecto de fin de carrera. Inicialmente, Pedro Latorre había explicado en qué consistía la idea inicial del proyecto en una reunión previa. En esta ocasión, se profundizó más en la idea y se indicaron unas pautas a seguir para investigar el estado del arte en esta área y plantear unos objetivos a cumplir durante este primer sprint.

Los objetivos de la reunión del primer sprint fueron: crear un repositorio en GitHub para el proyecto, probar dos métodos de procesamiento de imágenes, específicamente el Análisis de Componentes Principales (PCA) y la Teoría Retinex, y guardar la imagen resultante permitiendo acceder al valor de cada píxel. También se acordó investigar y aplicar métodos de agrupamiento de datos, como K-Means, en la imagen original y en la imagen procesada para comparar los resultados.

Además, se decidió eliminar el cuarto canal en las imágenes que lo tuvieran y evaluar críticamente los resultados obtenidos. Se enfatizó que no se esperaran resultados inmediatos, sino que se debía evaluar y descartar posibles caminos de manera progresiva.

En este momento del proyecto, simplemente se crearon una serie de tareas a completar antes de la próxima semana, por lo que no se tiene un registro temporal en Zube de cuándo se completó cada tarea. Más tarde, estas tareas se pasaron a GitHub el 26/03/2024 por recomendación del tutor, ya que así se podía reflejar el trabajo realizado hasta el momento. Estas issues tienen en el nombre [S1].

\subsection{Sprint 1 (11/03/2024 - 25/03/2024)}\label{sprint-1}

Este sprint tuvo como objetivos: mostrar la representación del histograma de los valores de los píxeles de cada canal en la imagen resultante, determinar los valores mínimos y máximos de los píxeles, y visualizar la matriz resultante utilizando a función de MATLAB `imshow'. Se discutió que, aunque `imshow(M, [])' normaliza la imagen para visualización, `imwrite' guarda la imagen sin normalizar. Se acordó guardar la imagen utilizando el método `saveas' para conservar el rango dinámico y almacenar la matriz en un archivo `.mat' con el resultado de la transformación. También se destacó la importancia de guardar una imagen que se pueda visualizar fácilmente y familiarizarse con los resultados de cada método aplicado.

Se decidió añadir los métodos de agrupamiento Fuzzy C-Means y GMM, y probar otros métodos de agrupamiento que tengan en cuenta el contexto y no solo el valor del píxel. Además, se acordó buscar información sobre Geoffrey Machlachlan para profundizar en el tema.

Al igual que en el sprint 0, simplemente se crearon una serie de tareas a completar antes de la próxima semana, por lo que no se tiene un registro temporal en Zube y en GitHub. Las tareas fueron añadidas el 26/03/2024. Estas issues tienen en el nombre [S2].

\subsection{Sprint 2 (25/03/2024 - 04/04/2024)}\label{sprint-2}

Este sprint tuvo como objetivo: trasladar los algoritmos de agrupamiento de Python a MATLAB, ya que los tiempos de ejecución en Python, el lenguaje utilizado hasta ese momento, eran muy elevados. Se decidió crear una comparación en la misma ventana, dependiendo del número de centros, para así poder observar las diferencias de manera más cómoda y rápida. Además, se acordó agregar 4-5 métodos más de transformación invariante.

También se decidió generar la imagen ground truth y evaluar el resultado mediante una máscara binaria, calculando la tasa de acierto con la fórmula  (nº de píxeles que coinciden / nº total de píxeles) * 100, y mostrar en la ventana el valor de esta tasa.

Al igual que en el sprint 0 y sprint 1, simplemente se crearon una serie de tareas a completar antes de la próxima semana, por lo que no se tiene un registro temporal en Zube. En GitHub, las tareas fueron añadidas el 26/03/2024. Estas issues tienen en el nombre [S3].

\subsection{Sprint 3 (04/04/2024 - 18/04/2024)}\label{sprint-3}

En la reunión de preparación de este sprint, Carlos López se unió como cotutor al proyecto a solicitud de Pedro Latorre. En el futuro, Carlos López pasaría a ser el tutor del proyecto de fin de grado.

Este sprint tuvo como objetivo familiarizarse con Overleaf y comenzar a rellenar las secciones de introducción, objetivos del proyecto y conceptos teóricos de la memoria del TFG. Además, se acordó crear una cuenta en Zube para organizar las tareas en milestones y reorganizar los directorios del proyecto. La nueva estructura incluirá un directorio `doc' con la plantilla de documentación de TFG disponible en UBUVirtual y un directorio para el programa MATLAB, que contendrá todos los métodos utilizados, separados en distintos scripts para mejorar la modularidad, reutilización y mantenibilidad del código.

Otro objetivo fue la integración de MATLAB con una aplicación web, iniciando un pequeño proyecto en GitHub para gestionar esta integración. También se planteó el desarrollo de una pequeña aplicación de escritorio de MATLAB, que servirá para probar distintos algoritmos sobre diferentes imágenes. De esta manera, además de demostrar numéricamente que hay una mejora notable tras la aplicación de métodos invariantes sobre piezas metálicas, permitirá probar distintos métodos sobre imágenes propias de forma sencilla y visual.

En el sprint anterior, no se logró encontrar un método de agrupamiento con contexto espacial que supusiera una mejora notable en los resultados, por lo que esta issue no se cerrará y formará parte también de este cuarto sprint.

Al ser este el primer sprint en el que se incorporó Zube para gestionar las issues abiertas, aún no se dominaba completamente la herramienta. Aunque ayudó con la gestión, no se creó el sprint en Zube, por lo que no se dispone de un gráfico burndown de este sprint.

\subsection{Sprint 4 (18/04/2024 - 02/05/2024)}\label{sprint-4}

Es en este sprint cuando se comienza a reflejar formalmente los distintos sprints en Zube. Gracias a esto, a partir de este momento, resultó más cómodo trabajar con las issues de cada sprint.

Este sprint tuvo como objetivos: mejorar las funcionalidades de la aplicación y realizar cambios menores en el texto mostrado al usuario dentro de la aplicación para evitar confusiones y corregir incoherencias. Se acordó cambiar el orden en el que se muestran los resultados de la aplicación y se discutió la importancia de eliminar las imágenes mostradas de ejecuciones anteriores.

También se añadió un selector para guardar los archivos y su correspondiente mensaje de confirmación indicando que los archivos se han guardado correctamente. Además, se buscó una solución para el problema relacionado con el sistema de directorios donde se guardan los archivos.

Finalmente, se planteó la creación de otra aplicación para poder probar con imágenes propias. Al tener una imagen ground truth, se podría obtener un porcentaje de acierto de la ejecución, lo que ayudaría a evaluar mejor los resultados obtenidos.

\imagen{sprint4}{Burndown del sprint 4.}

Como se puede ver en el gráfico burndown \ref{fig:sprint4}, las issues de este sprint no fueron marcadas como completadas hasta el último día. En ese momento, no se era consciente de que debían cerrarse en lugar de dejarlas en revisión para cerrarlas después de la reunión.


\subsection{Sprint 5 (02/05/2024 - 23/05/2024)}\label{sprint-5}

Este sprint tuvo como objetivo mejorar la experiencia del usuario y optimizar la funcionalidad de la aplicación. Se implemento una waitbar que periódicamente informaba sobre el estado de la ejecución, indicando que podría tomar mucho tiempo, además de añadir la funcionalidad de borrar resultados anteriores. También se realizaron pequeños cambios en la interfaz, como eliminar el `-' de los porcentajes y añadir un botón para guardar imágenes.

Se realizaron distintas modificaciones en la estructura de la interfaz, añadiendo menús de ayuda, una lista de nombres de imágenes en caché, un menú de bienvenida y ayuda contextual. Se actualizó el nombre de los archivos de resultados, añadiendo la fecha de ejecución al inicio. Además, se corrigieron errores en la selección de imágenes desde rutas diferentes a /data y se mejoró la gestión de errores.

Se llegó a la conclusión de agregar una caché de imágenes para almacenar temporalmente los resultados, cargando ejecuciones anteriores con los mismos parámetros para reducir el coste computacional. Al cerrar la aplicación, la caché se borraría. Finalmente, se planteó la corrección de algunos errores detectados en los algoritmos, así como una mejora de la documentación y el README del proyecto.

\imagen{sprint5}{Burndown del sprint 5.}

Como se puede ver en el gráfico burndown \ref{fig:sprint5}, las issues de este sprint no fueron marcadas como completadas hasta el último día. En ese momento, no se era consciente de que debían cerrarse en lugar de dejarlas en revisión para cerrarlas después de la reunión.


\subsection{Sprint 6 (23/05/2024 - 30/05/2024)}\label{sprint-6}

Este sprint tuvo como objetivo mejorar la funcionalidad y usabilidad de la aplicación. Se planean diversas mejoras en la documentación en HTML y la creación de un README en la carpeta de imágenes. Además, se modificará la visualización de los nombres de las imágenes dentro de la aplicación y se añadirán tooltips para una mejor comprensión del usuario.

Se establecerán condiciones numéricas en los campos de número de centros y se añadirá un límite máximo de iteraciones en los métodos de agrupamiento. También se trabajará en la caché para hacerla persistente, permitiendo borrar el historial y mejorando su escalabilidad.

En cuanto a la evaluación de la calidad, se incorporará una medida de comparación utilizando la imagen ground truth y se mejorará la nomenclatura de los algoritmos invariantes, reemplazando los nombres vacíos con “no aplicado”. Se abordarán problemas técnicos como la minimización de la app al cargar una imagen, la eliminación de la imagen ground truth al introducir una nueva que no tenga una asociada, y se revisarán soluciones para evitar que la aplicación se detenga por errores. Además, se tratará de introducir concurrencia para manejar tareas como cancelar la ejecución.

\imagen{sprint6}{Burndown del sprint 6.}

Como se puede ver en el gráfico burndown \ref{fig:sprint6}, las issues de este sprint se fueron cerrando correctamente a medida que se completaban.


\subsection{Sprint 7 (31/05/2024 - 10/06/2024)}\label{sprint-7}

Este sprint tuvo como objetivo preparar la primera release oficial de la aplicación. Se realizarán mejoras en la gestión de errores para asegurar una experiencia de usuario más robusta. Además, se tratará de aplicar concurrencia para permitir la cancelación de la ejecución, una tarea que no se logró completar en el sprint anterior.

Se deshabilitarán los botones durante la ejecución de procesos para evitar acciones no deseadas y se revisará el código, actualizando la documentación interna para mantener la coherencia. También se implementará soporte bilingüe, añadiendo opciones en español e inglés para hacer la aplicación accesible a un público más amplio.

\imagen{sprint7}{Burndown del sprint 7.}

Como se puede ver en el gráfico burndown \ref{fig:sprint7}, las issues de este sprint se fueron cerrando correctamente a medida que se completaban, a excepción de una que, por falta de tiempo, quedó pendiente para completar en el siguiente sprint.


\subsection{Sprint 8 (11/06/2024 - 27/06/2024)}\label{sprint-8}

En este sprint, se implementaron mejoras y se preparó la aplicación para su versión oficial. Se mejoraron los tooltips, especialmente el de centros, y se añadió la opción de imprimir y visualizar imágenes en la caché, además de modificar el botón `Abrir directorio' por `Ver imágenes'. Se habilitó la ordenación por filas en la caché y se actualizó la imagen de bienvenida.

Para garantizar compatibilidad, la aplicación se adaptó a Linux, y se realizaron pruebas unitarias. Se creó un README detallado y se reorganizaron los directorios, eliminando los obsoletos. También se subió la versión anterior con mejoras en el manejo de errores y adaptación al inglés.

En la documentación, se añadieron citas y se revisaron correcciones en la memoria, y se redactaron el plan de proyecto, especificaciones de requisitos y diseño en los anexos.

Finalmente, se realizó la versión con todos los cambios del sprint, consolidando las mejoras para la versión final.

\imagen{sprint8}{Burndown del sprint 8.}

Como se puede apreciar en la figura \ref{fig:sprint8}, no se lograron finalizar todas las tareas planificadas para este supuesto sprint final, debido a que la documentación requirió una considerable cantidad de tiempo. Por este motivo, se acordó reunirse a mitad de semana con todas estas tareas finalizadas y con al menos una versión inicial de la documentación completada.

\subsection{Sprint 9 (28/06/2024 - 01/07/2024)}\label{sprint-9}

Este sprint se dedicó completamente a finalizar las tareas pendientes del periodo anterior, incluyendo la elaboración de una primera versión de la documentación, tanto de la memoria como de los anexos. Además, se identificó y resolvió un error en la aplicación.

\subsection{Resumen del tiempo empleado}\label{resumen-del-tiempo-empleado}

A continuación se muestra un resumen del tiempo dedicado a la realización del proyecto.

\begin{table}[h!]
    \centering
    \begin{tabular}{c c}
        \hline
        \textbf{Categoría}      & \textbf{Tiempo (horas)} \\
        \hline
        \textbf{Sprint 0}       & 20    \\
        \textbf{Sprint 1}       & 32    \\
        \textbf{Sprint 2}       & 20    \\
        \textbf{Sprint 3}       & 26    \\
        \textbf{Sprint 4}       & 49    \\
        \textbf{Sprint 5}       & 54    \\
        \textbf{Sprint 6}       & 28    \\
        \textbf{Sprint 7}       & 42    \\
        \textbf{Sprint 8}       & 71    \\
        \textbf{Sprint 9}       & 22    \\
        \hline
        \textbf{Total}          & \textbf{364 horas}\\
        \hline
    \end{tabular}
    \caption{Desglose de tiempo dedicado al proyecto.}\label{tabla:tiempo-dedicado-al-proyecto}
\end{table}

\section{Estudio de viabilidad}\label{estudio-de-viabilidad}

A continuación, se realizará tanto el estudio de viabilidad económica como el de viabilidad legal.

\subsection{Viabilidad económica}\label{viabilidad-economica}

En este apartado se analizarán los costes y beneficios estimados de haber desarrollado el proyecto en un entorno empresarial. Cabe mencionar que dichos datos son aproximados.

\subsubsection{Costes}\label{costes}

A continuación, se realizará tanto el análisis de costes materiales y beneficios.

\textbf{Costes materiales:}

En este apartado se desglosarán todos aquellos costes materiales, tanto de hardware como de software, necesarios para el correcto desarrollo del proyecto. En ambos casos, se considera que se amortizarán a 5 años y se han utilizado durante los cuatro meses de desarrollo del proyecto.

El desglose de todos los dispositivos hardware necesarios para la realización del proyecto se encuentra a continuación en la figura \ref{tabla:costes-de-hardware}.

\begin{table}[h!]
    \centering
    \begin{tabular}{l l l}
        \hline
        \textbf{Concepto}               & \textbf{Coste}    & \textbf{Coste amortizado} \\
        \hline
        \textbf{Ordenador portátil}     & 900,00€           & 60,00€                    \\
        \hline
        \textbf{Total}                  & \textbf{900,00€}  & \textbf{60,00€}           \\
        \hline
    \end{tabular}
    \caption{Costes de hardware.}\label{tabla:costes-de-hardware}
\end{table}

En cuanto a las licencias software no gratuitas se tiene el siguiente desglose en la figura \ref{tabla:costes-de-software}.

\begin{table}[h!]
    \centering
    \begin{tabular}{l l l}
        \hline
        \textbf{Concepto}                   & \textbf{Coste}        & \textbf{Coste amortizado} \\
        \hline
        \textbf{Windows 10 Pro}             & 259,00€               & 17,27€                    \\
        \textbf{MATLAB Academic (Anual)}    & 262,00€               & 87,33€                    \\
        \hline
        \textbf{Total}                      & \textbf{521,00€}      & \textbf{104,60€}          \\
        \hline
    \end{tabular}
    \caption{Costes de software.}\label{tabla:costes-de-software}
\end{table}

Hay que destacar que la licencia de MATLAB utilizada ha sido la Academic que la propia Universidad de Burgos ofrece a los alumnos, la cual se paga anualmente.

\textbf{Costes de personal:}

El proyecto lo ha llevado a cabo un ingeniero informático recién graduado a tiempo completo durante cuatro meses. Se considera que el salario bruto anual es de 24.000,00€ \cite{sueldoIIespaña}, el IRPF sobre el salario bruto es del 13,55\% y la Seguridad Social es del 6,35\% respectivamente como se puede ver en la figura \ref{tabla:costes-de-personal}. Se considera que la cantidad de horas de trabajo anuales en España a tiempo completo es de 1820 \cite{horasAnuales}.

\begin{table}[h!]
    \centering
    \begin{tabular}{l l}
        \hline
        \textbf{Concepto}                   & \textbf{Coste}        \\
        \hline
        \textbf{Salario anual en bruto}     & 24.000,00€            \\
        \textbf{IRPF}                       & 3.251,04€             \\
        \textbf{Seguridad Social}           & 1.524,00€             \\
        \textbf{Salario anual neto}         & 19.224,96€            \\
        \textbf{Salario neto por hora}      & 10,56€                \\
        \hline
        \textbf{Salario 364 horas}          & \textbf{3.844,99€}    \\
        \hline
    \end{tabular}
    \caption{Costes de personal.}\label{tabla:costes-de-personal}
\end{table}

\textbf{Costes totales:}

El sumatorio de todos los distintos tipos de costes se encuentra en la figura \ref{tabla:costes-de-totales}.

\begin{table}[h!]
    \centering
    \begin{tabular}{l l}
        \hline
        \textbf{Concepto}   & \textbf{Coste}        \\
        \hline
        \textbf{Hardware}   & 60,00€                \\
        \textbf{Software}   & 104,60€               \\
        \textbf{Personal}   & 3.844,99€             \\
        \hline
        \textbf{Total}      & \textbf{4.009,59€}    \\
        \hline
    \end{tabular}
    \caption{Costes totales.}\label{tabla:costes-de-totales}
\end{table}

\subsubsection{Beneficios}\label{beneficios}

La aplicación desarrollada se distribuirá de forma libre y sin restricciones, permitiendo su uso sin limitaciones adicionales. La única condición necesaria para su funcionamiento será disponer de MATLAB. En una fase posterior, se evaluará la posibilidad de incorporar características adicionales que mejoren la experiencia del usuario, manteniendo siempre el enfoque en la accesibilidad y la libertad de uso.

\subsection{Viabilidad legal}\label{viabilidad-legal}

Apartado dedicado al marco legal que envuelve el proceso de desarrollo del proyecto, su distribución, modificación y utilización. No es necesario atender al uso de licencias de terceros, pues todo el proyecto ha sido desarrollado con la licencia MATLAB Academic proporcionada por la Universidad de Burgos para desarrolladores individuales o estudiantes y bajo una finalidad no comercial. En concreto, la licencia utilizada ha sido:

\begin{itemize}
  \item \textbf{MATLAB Academic:} Al tratarse de un desarrollador individual, el uso de la licencia MATLAB Academic está justificado. Esta licencia es distribuida de forma gratuita por la Universidad de Burgos a sus alumnos.
\end{itemize}

Debido al uso de distintos papers y toolboxes de MATLAB la aplicación esta limitada por las licencias de estos. Las licencias son las siguientes:

\newpage

\tabla{Licencias}{p{0.56\columnwidth}|p{0.39\columnwidth}}{2}{licencias}{
  \textbf{Elemento} & \textbf{Licencia} \\
}{
  Statistics and Machine Learning Toolbox & MathWorks Software License Agreement (MWL) \\
  Fuzzy Logic Toolbox & MathWorks Software License Agreement (MWL) \\
  MATLAB Compiler & MathWorks Software License Agreement (MWL) \\
  HMRF-EM-image & BSD 3-Clause License \\
  Road Detection Based on Illuminant Invariance & IEEE Terms and Conditions \\
  Illumination Invariant Imaging: Applications in Robust Vision-based Localisation, Mapping and Classification for Autonomous Vehicles & IEEE Terms and Conditions \\
  Visual Road Following Using Intrinsic Images & IEEE Terms and Conditions \\
  Lighting Invariant Urban Street Classification & IEEE Terms and Conditions \\
  PCA based Computation of Illumination-Invariant Space for Road Detection & IEEE Terms and Conditions \\
}

El proyecto actual ha sido asignado bajo la licencia BSD 3-Clause. Esto implica que cualquier usuario es libre de compartir, redistribuir y adaptar la aplicación, siempre y cuando se cumplan las siguientes condiciones:

\begin{itemize}
  \item \textbf{Atribución:} Se debe dar crédito apropiado al autor original, manteniendo el aviso de copyright, la lista de condiciones y la exención de responsabilidad en todas las distribuciones.
  \item \textbf{Redistribución:} Las redistribuciones en formato de código fuente deben incluir el aviso de copyright, esta lista de condiciones y la exención de responsabilidad.
  \item \textbf{Restricción de Uso de Nombre:} Ni el nombre de los autores ni los nombres de sus colaboradores pueden ser utilizados para apoyar o promocionar productos derivados sin un permiso específico por escrito.
\end{itemize}

Además, el uso de MATLAB Runtime y ciertas herramientas de MATLAB en este proyecto está sujeto al MathWorks Software License Agreement (MWL). Los usuarios deben tener una licencia válida de MathWorks para ejecutar este software. La redistribución de MATLAB Runtime o cualquier toolbox de MathWorks no está permitida. Los usuarios deben descargar e instalar MATLAB Runtime desde el sitio oficial de MathWorks de acuerdo con los términos de su licencia segun se indica en \url{https://www.mathworks.com/company/aboutus/policies_statements/}.

Esta combinación de licencias asegura que la aplicación mantenga su accesibilidad y flexibilidad, fomentando la colaboración y el desarrollo comunitario mientras protege los derechos del autor original y cumple con las restricciones de las herramientas utilizadas.

