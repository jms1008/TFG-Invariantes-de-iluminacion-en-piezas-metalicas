\apendice{Anexo de sostenibilización curricular}

\section{Introducción}\label{introduccion-ods}

Este anexo tiene como objetivo reflexionar sobre los aspectos de sostenibilidad abordados en el proyecto. Durante el desarrollo de este, se han adquirido competencias en sostenibilidad que se aplican en el Trabajo de Fin de Grado. A continuación, se detallan estos aspectos y competencias adquiridas.

\section{Desarrollo del proyecto}\label{desarrollo-del-proyecto-ods}

El proyecto consistió en investigar, probar y finalmente crear una aplicación que utiliza transformaciones invariantes para mejorar la segmentación de imágenes de piezas metálicas. La hipótesis inicial era que la incorporación de estas transformaciones aumentaría la precisión de los algoritmos de segmentación, una suposición que fue confirmada a través de experimentos rigurosos.

Primero, se realizó una extensa revisión bibliográfica sobre técnicas de transformación invariante y segmentación de imágenes. Se analizaron diversas metodologías y se seleccionaron las que mostraban mayor potencial para mejorar los resultados de segmentación. Esta fase inicial fue crucial para establecer una base sólida y asegurar que las técnicas seleccionadas fueran las más adecuadas para el objetivo del proyecto.

Posteriormente, se desarrollaron prototipos de la aplicación, implementando diferentes transformaciones invariantes y evaluando su impacto en la segmentación. Se utilizó un conjunto de imágenes variado para garantizar que los resultados fueran robustos y aplicables a la industria, teniendo en cuenta diferentes niveles de iluminación y complejidad de las imágenes. Los resultados mostraron un incremento significativo en la tasa de acierto, confirmando la validez de la hipótesis.

\section{Competencias de sostenibilidad adquiridas}\label{competencias-de-sostenibilidad-adquiridas-ods}

Durante el desarrollo del proyecto, se adquirieron diversas competencias de sostenibilidad, aplicadas de manera transversal a lo largo del trabajo:

\subsection{Contextualización crítica del conocimiento}\label{contextualización-crítica-del-conocimient-ods}

Se desarrolló la capacidad de contextualizar críticamente el conocimiento adquirido, relacionándolo con problemáticas sociales y ambientales. La mejora en la segmentación de piezas metálicas tiene aplicaciones directas en la industria, donde una segmentación precisa es crucial para reducir el desperdicio de materiales y mejorar la eficiencia de los procesos de producción, contribuyendo así a una menor huella ambiental.

\subsection{Uso sostenible de recursos}\label{uso-sostenible-de-recursos-ods}

El proyecto promovió el uso sostenible de recursos tecnológicos y materiales. Se optimizaron los algoritmos para que fueran eficientes en términos de tiempo de computación y consumo de energía, reduciendo así el impacto ambiental. Además, al mejorar la segmentación y la inspección de piezas metálicas, se minimiza el desperdicio de materiales y se incrementa la vida útil de los productos, promoviendo un ciclo de vida más sostenible para los recursos metálicos.

\subsection{Aplicación de principios éticos}\label{aplicación-de-principios-éticos-ods}

Se aplicaron principios éticos relacionados con la sostenibilidad en cada etapa del proyecto. Esto incluyó la elección de técnicas y métodos que minimizaran el impacto ambiental y social negativo. Además, se aseguró la transparencia y la ética en la presentación de resultados y la difusión de conocimientos, promoviendo un uso responsable y sostenible de la tecnología desarrollada.

\section{Reflexión personal}\label{reflexión-personal-ods}

Este proyecto ha sido una oportunidad para integrar competencias de sostenibilidad en un contexto técnico, demostrando que la innovación tecnológica puede y debe alinearse con principios de desarrollo sostenible. La aplicación de transformaciones invariantes no solo mejora la precisión técnica, sino que también puede contribuir a avances significativos en la industria, reduciendo el desperdicio de materiales y mejorando la eficiencia energética.

La experiencia estos últimos años ha resaltado la importancia de considerar siempre el impacto ambiental y social de nuestras decisiones técnicas. En el futuro, como profesionales, es esencial que adoptemos un enfoque responsable y sostenible en nuestro trabajo, contribuyendo así a la creación de un futuro más equitativo y sostenible.

En conclusión, este proyecto no solo ha demostrado la validez de los métodos de transformación invariante aplicados a piezas metálicas, sino que también ha sido un ejercicio práctico en la aplicación de principios de sostenibilidad, demostrando que es posible desarrollar tecnología de manera ética y responsable. La integración de transformaciones invariantes en la segmentación de piezas metálicas representa un avance significativo tanto en la eficiencia de la producción industrial como en la minimización del impacto ambiental, subrayando la importancia de la sostenibilidad en la ingeniería y la tecnología.
