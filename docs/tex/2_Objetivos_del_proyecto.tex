\capitulo{2}{Objetivos del proyecto}

% TExto

\section{Objetivos generales}\label{objetivos-generales}

\begin{itemize}
    \tightlist
        \item
            Demostrar que al aplicar métodos de transformación invariante, se consiguen mejores resultados a la hora de identificar piezas metálicas mediante métodos de segmentación de imágenes.
        \item
            Desarrollar una aplicación para Windows que permita al usuario identificar la pieza de una imagen que introduzca ofreciendo diferentes algoritmos tanto de transformación invariante como de segmentación de imágenes.
        \item
            Desarrollar una aplicación para Windows que permita mostrar la mejoría que suponen los métodos de transformación invariante mostrando una comparación de los resultados tanto visual como numérica.
        \item
            Guardar los resultados tanto de la imagen original, la resultante del método de transformación invariante como de las correspondientes imágenes tras la segmentación.
\end{itemize}

\section{Objetivos técnicos}\label{objetivos-tecnicos}

\begin{itemize}
    \tightlist
        \item
            Investigar el estado del arte de los métodos de invariación de iluminación y como aplicarlos a piezas metálicas.
        \item
            Desarrollar una aplicación que permita aplicar sobre una imagen los distintos métodos y guardar los resultados.
        \item
            Utilizar la plataforma GitHub como sistema de control de versiones
        \item
            Utilizar Zube como herramienta de gestión de proyectos.
        \item
            Realizar tests... .
        \item
            Utilizar un sistema de documentación... .
        \item
            Hostear una web app... .
\end{itemize}

\section{Objetivos personales}\label{objetivos-personales}

\begin{itemize}
    \tightlist
        \item
            Realizar una aportación a la modernización de la industria.
        \item
            Profundizar en el uso de MATLAB y el cual tiene una gran potencia computacional, destacando en cálculos matriciales.
        \item
            Explorar métodos de visión artificial.
        \item 
            Facilitar sistemas escalables en diversos entornos.
\end{itemize}