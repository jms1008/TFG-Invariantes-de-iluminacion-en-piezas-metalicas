\capitulo{7}{Conclusiones y Líneas de trabajo futuras}

En esta sección se presentan las conclusiones obtenidas del proyecto, así como las posibles lineas de trabajo para futuras investigaciones que podrían dar continuidad al proyecto.

\section{Concusiones}\label{conclusiones}

Tras la finalización del proyecto, se puede concluir que el objetivo principal de demostrar la mejora en la identificación de piezas en imágenes mediante el uso de algoritmos de transformación invariante ha sido alcanzado exitosamente. Este proyecto ha requerido la aplicación exhaustiva de una gran parte de los conocimientos adquiridos a lo largo de la carrera, desde fundamentos teóricos hasta habilidades prácticas en programación y análisis de datos.

La búsqueda y análisis del estado del arte han sido aspectos cruciales en el desarrollo del proyecto. Este proceso no solo ha permitido identificar los algoritmos más prometedores, sino que también ha fomentado el desarrollo de un pensamiento crítico esencial para la investigación científica. La capacidad de evaluar, seleccionar y descartar algoritmos basándose en sus resultados y eficiencia ha sido un aprendizaje significativo. Se han probado diversas técnicas, evaluando su rendimiento en diferentes escenarios y ajustando los parámetros necesarios para optimizar los resultados.

Además, la gestión del tiempo ha sido un aspecto en el que se ha observado una notable mejora. Al enfrentarse a la planificación y ejecución de tareas complejas, se ha demostrado la importancia de un conocimiento profundo del tema para estimar adecuadamente los tiempos requeridos. Esta experiencia ha permitido desarrollar habilidades de gestión del tiempo más precisas y efectivas, cruciales para la realización de proyectos futuros.

El desarrollo de una aplicación que permite a los usuarios aplicar estos algoritmos de manera sencilla representa un logro significativo y un valor añadido al proyecto. Esta aplicación no solo facilita la accesibilidad y usabilidad de los algoritmos propuestos, sino que también proporciona una herramienta práctica que puede ser utilizada en diversos contextos industriales y de investigación. La interfaz amigable y la funcionalidad robusta de la aplicación aseguran que incluso usuarios con conocimientos técnicos limitados puedan beneficiarse de los avances logrados.

En resumen, el proyecto ha cumplido con sus objetivos iniciales y ha proporcionado importantes aprendizajes tanto a nivel técnico como de gestión. La combinación de investigación teórica, desarrollo práctico y aplicación real ha resultado en una experiencia integral que ha contribuido significativamente a la formación profesional y académica. Los conocimientos y habilidades adquiridos a lo largo de este proceso serán de gran utilidad en futuros desafíos profesionales.

\section{Líneas de trabajo futuras}\label{líneas-de-trabajo-futuras}

Tras un análisis detallado de las posibles líneas a seguir, se han identificado varias mejoras que podrían ser implementadas en el futuro. Estas mejoras no se han podido realizar dentro del marco de tiempo del proyecto debido a la complejidad y el esfuerzo adicional que requieren. A continuación, se detallan estas líneas de trabajo futuras:

\subsection{Mejora de algoritmos}\label{mejora-de-algoritmos}

En esta sección se describen los posibles avances en la investigación, implementación y optimización de los algoritmos actuales:

\begin{itemize}
    \item Mejora del segmentado: Ampliar las capacidades del algoritmo de segmentación para permitir una diferenciación más precisa, no solo entre la pieza y el fondo, sino también identificando las zonas fresadas de la pieza. Esta mejora permitiría una mayor precisión en la identificación de detalles específicos de las piezas, aumentando la exactitud del reconocimiento.
    \item Identificación en imágenes complejas: optimizar los algoritmos para mejorar la identificación de piezas específicas en imágenes de gran complejidad. Esto incluye la capacidad de reconocer piezas en entornos con múltiples objetos y fondos variados, facilitando su utilización en situaciones reales.
\end{itemize}

\subsection{Mejoras de experiencia de usuario}\label{mejoras-de-experiencia-de-usuario}

Esta sección aborda las posibles mejoras relacionadas con la funcionalidad y la interfaz de usuario de la aplicación desarrollada:

\begin{itemize}
    \item Soporte para dispositivos con sistema operativo Mac: ampliar la compatibilidad de la aplicación para incluir dispositivos que operen bajo el sistema operativo Mac. Esto permitirá a un mayor número de usuarios acceder y utilizar la herramienta, aumentando su alcance y utilidad.
    \item Desarrollo de una web app: crear una aplicación web que permita a los usuarios acceder de manera más cómoda a un espacio de trabajo personal. Esta plataforma en línea ofrecería una mayor flexibilidad y accesibilidad, permitiendo a los usuarios trabajar desde cualquier lugar y dispositivo con acceso a internet.
    \item Implementación de concurrencia en la aplicación: mejorar la eficiencia de la aplicación mediante la implementación de concurrencia. Esto permitiría la ejecución simultánea de múltiples tareas, mejorando el rendimiento y reduciendo los tiempos de espera para los usuarios además de permitir cancelar la ejecución en cualquier momento.
\end{itemize}
