\capitulo{1}{Introducción}

En el ámbito de la fabricación, las piezas metálicas son fundamentales para una amplia variedad de aplicaciones industriales. La precisión en su fabricación es crucial para garantizar la funcionalidad y durabilidad de los productos finales. 

El control de calidad en el proceso de producción es de gran importancia, ya que asegura que los productos finales cumplan con los estándares establecidos y satisfagan las expectativas del cliente. A través de técnicas de inspección y pruebas rigurosas, se pueden identificar y corregir defectos en etapas tempranas del proceso de producción. Esto no solo reduce el riesgo de fallos en el producto terminado, sino que también optimiza el uso de materiales y recursos, incrementando la eficiencia general.\cite{jmmp6060141}

Comúnmente, esta inspección técnica y pruebas suelen ser llevadas a cabo por un operario, ya que los sistemas basados en la toma de imágenes para detectar fallos en el producto dependen en gran medida de la calidad de la fotografía tomada. En la mayoría de las ocasiones, a menos que el fondo sea muy fácilmente distinguible de la pieza y las condiciones de iluminación sean ideales, los métodos de agrupamiento difícilmente pueden identificar la pieza correctamente. Por esta razón, los métodos de transformación invariante desempeñan un papel esencial, ya que permiten compensar y corregir las distorsiones causadas por variaciones en las condiciones de iluminación, como reflejos y sombras los cuales en piezas metálicas se dan con mucha facilidad. \cite{vafadar2021}\cite{jmmp6060141}

Los métodos invariantes son responsables de generar una nueva imagen a partir de la original en la cual las reflexiones de luz, las sombras o errores inducidos por las condiciones lumínicas en el momento de la captura de imágenes estén mitigados. Al aplicar sobre esta nueva imagen algún método de agrupamiento de imágenes, se obtendrá un resultado mejor, o dicho en otras palabras, más cercano a la realidad.

Un método de agrupamiento de imágenes sirve para segmentarlas en diferentes partes, identificando y clasificando áreas similares dentro de la imagen. Este proceso es crucial para separar la pieza metálica del fondo y otros elementos no deseados, facilitando así una inspección técnica más precisa.\cite{bento2019non}

La implementación de estos métodos de manera previa a la segmentación de imágenes no solo mejora la precisión, sino que también reduce la variabilidad que se da por diferencias en la iluminación del entorno. Esto es particularmente importante en entornos de producción industrial, donde las condiciones de iluminación pueden no ser siempre ideales y variar significativamente, afectando la visibilidad y el contraste en las imágenes capturadas.\cite{karunakaran2012rapid}

En resumen, el uso de métodos de transformación invariante en la segmentación de imágenes de piezas metálicas tiene un papel muy importante a la hora de contrarrestar los efectos adversos de las condiciones de iluminación. Estas técnicas no solo mejoran considerablemente la precisión de los resultados y la consistencia en la correcta identificación de las piezas metálicas en imágenes, sino que permiten automatizar el control de calidad dentro de la propia la industria, optimizando así los procesos de producción. \cite{jmmp6060141}


%-------------------------------------------------------------------------------------------
%
% Lo siguiente lo he tomado de la estructura del proyecto de go-bees ya que me parece que esta bastante bien estructurado, por lo que aunque en un futuro lo tenga que cambiar ya que esta literalmente copiado para hacerme una idea de como estructurar esta parte, no es mio.
% https://github.com/davidmigloz/go-bees
%
%-------------------------------------------------------------------------------------------

\section{Estructura de la memoria}\label{estructura-de-la-memoria}

La memoria sigue la siguiente estructura:

\begin{itemize}
    \tightlist
        \item
            \textbf{Introducción:} breve descripción del problema a resolver y la solución propuesta. Estructura de la memoria y listado de materiales adjuntos.
        \item
            \textbf{Objetivos del proyecto:} exposición de los objetivos que persigue el proyecto.
        \item
            \textbf{Conceptos teóricos:} breve explicación de los conceptos teóricos clave para la comprensión de la solución propuesta.
        \item
            \textbf{Técnicas y herramientas:} listado de técnicas metodológicas y herramientas utilizadas para gestión y desarrollo del proyecto.
        \item
            \textbf{Aspectos relevantes del desarrollo:} exposición de aspectos destacables que tuvieron lugar durante la realización del proyecto.
        \item
            \textbf{Trabajos relacionados:} estado del arte en el campo de la monitorización de la actividad de vuelo de colmenas y proyectos relacionados.
        \item
            \textbf{Conclusiones y líneas de trabajo futuras:} conclusiones obtenidas tras la realización del proyecto y posibilidades de mejora o expansión de la solución aportada.
\end{itemize}

Junto a la memoria se proporcionan los siguientes anexos:

\begin{itemize}
    \tightlist
        \item
            \textbf{Plan del proyecto software:} planificación temporal y estudio de viabilidad del proyecto.
        \item
            \textbf{Especificación de requisitos del software:} se describe la fase de análisis; los objetivos generales, el catálogo de requisitos del sistema y la especificación de requisitos funcionales y no funcionales.
        \item
            \textbf{Especificación de diseño:} se describe la fase de diseño; el ámbito del software, el diseño de datos, el diseño procedimental y el diseño arquitectónico.
        \item
            \textbf{Manual del programador:} recoge los aspectos más relevantes relacionados con el código fuente (estructura, compilación, instalación, ejecución, pruebas, etc.).
        \item
            \textbf{Manual de usuario:} guía de usuario para el correcto manejo de la aplicación.
\end{itemize}

\section{Materiales adjuntos}\label{materiales-adjuntos}

Los materiales que se adjuntan con la memoria son: 

\begin{itemize}
    \tightlist
        \item
            Aplicación de escritorio para probar tanto con imágenes propias como con algunas de ejemplo.
        \item
            Proyecto de MATLAB para poder probar tanto con imágenes propias como con algunas de ejemplo además de tener acceso al código para añadir o modificar los algoritmos que se utilizan.
\end{itemize}

Además, los siguientes recursos están accesibles a través de internet:

\begin{itemize}
    \tightlist
        \item
            Repositorio del proyecto. \href{https://github.com/jms1008/TFG-Invariantes-de-iluminacion-en-piezas-metalicas}{GitHub}
\end{itemize}
